\chapter{Conclusions}
\label{ch:conclusions}

The presented results show that constrained adversarial networks enable the encoding of useful prior knowledge in the game theoretic scenario initially proposed by generative adversarial networks. The conducted experiments reveal how constraints on discrete domains can be introduced in the learning procedure to instil knowledge in the deep generative model. The effect is that the proposed architecture is able to approximate the data distribution more closely than its constraints-unaware counterpart of boundary-seeking generative adversarial networks.

Furthermore, by exploring different design choices and analysing several factors, this work offers a number of insights on how these may individually impact on the final performance of the model. For instance, it is observed that the improvement over the baseline seems to increase proportionally with the amount of information conveyed by constraints and becomes particularly relevant when training data is not abundant.

The results also show that the knowledge instillation process is effective even if the information is indirectly provided to the generator via penalty functions computed by the discriminator. Evaluating other design choices for the model is left for future work. Many other architectures, in principle, could provide equivalent or better results. For instance, the original adversarial model may be extended to introduce a third constraints-aware network acting as a regulariser.

Further research is necessary to generalize our results to a wider variety of domains and constraints types. The synthetic data sets used in our experimental setup only allow some preliminary considerations and further testing is required on real-world data. However, the encouraging empirical results indicate a strong potential of constrained adversarial networks to be useful in many other contexts involving constraints beside the obvious one of object generation. For instance, due to its performance in generating new data from input noise, constrained adversarial network could become a building block for larger systems addressing more complex problems, such as optimization, or with specific requirements, such as uniform solution sampling.
